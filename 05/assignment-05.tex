\title{CS 726 Assignment 5}
\author{Ruochen Lin}
\documentclass[11pt]{article}
\usepackage{amsmath,amsfonts,amssymb,amsthm}
\usepackage{mathtools}
\usepackage{commath}
\begin{document}
\maketitle
\section{}
With 
\begin{equation}\begin{split} 
p^C &= -\frac{g^Tg}{g^TBg}g, \\
p^B &= -B^{-1}g, \\
\gamma &= \frac{(g^Tg)^2}{(g^TBg)(g^TB^{-1}g)}
\end{split}\nonumber\end{equation} 
and $\bar\gamma\in(\gamma,1]$, we can devise the following function to characterize $p$ in the three 'legs':
\begin{equation}\begin{split} 
\tilde p(\tau)=\left\{
\begin{array}{rcl}
&\tau p^C & {\tau \in (0,1]}\\
&p^C + (\tau-1)(\tilde\gamma p^B - p^C) & {\tau \in (1,2]}\\
&\tilde \gamma p^B + (\tau - 2) (p^B - \tilde\gamma p^B) & {\tau \in (2,3]}
\end{array} \right.
\end{split}\nonumber\end{equation}
As $p$ moves along the path, $\tau$ increases from $0$ to $3$. \\[0.3cm]
It's trivial to show that $\norm{\tilde p}$ increases with $\tau$ in the first and third segments of the path, because $\tilde p$ is simply moving from the origin to $p^C$ and from $\tilde\gamma p^B$ to $p^B$, respectively. In the second leg, if we define $\alpha = \tau - 1$ for convenience, we have 
\begin{equation}\begin{split} 
\frac{\partial \norm{\tilde p}^2}{\partial \alpha} &= 2\tilde p^T \frac{\partial \tilde p}{\partial\alpha}\\
&= 2[P^C + \alpha (\tilde\gamma p^B-p^C)]^T(\tilde\gamma p^B - p^C) \\
&=2(p^C)^T(\tilde \gamma p^B-p^C) + 2\alpha\norm{\tilde\gamma p^B - p^C}^2,
\end{split}\nonumber\end{equation} 
in which $2\alpha\norm{\tilde\gamma p^B - p^C}^2\geq 0$ and 
\begin{equation}\begin{split} 
(p^C)^T(\tilde\gamma p^B-p^C) & = -\frac{g^Tg}{g^TBg}g^T\Big( -\tilde\gamma B^{-1}g + \frac{g^Tg}{g^TBg}g \Big)\\
&= \frac{(g^Tg)(g^TB^{-1}g)}{g^TBg}\tilde\gamma - \frac{(g^Tg)^3}{(g^TBg)^2}\\
&> \frac{(g^Tg)(g^TB^{-1}g)}{g^TBg}\cdot \frac{(g^Tg)^2}{(g^TBg)(g^TB^{-1}g)} - \frac{(g^Tg)^3}{(g^TBg)^2}\\
&=\frac{(g^Tg)^3}{(g^TBg)^2} - \frac{(g^Tg)^3}{(g^TBg)^2}\\
&=0 \\
\Rightarrow &\frac{\partial\norm{\tilde p}^2}{\partial \alpha} >0,
\end{split}\nonumber\end{equation} 
in the second step of which we used $\tilde \gamma > \gamma = \frac{(g^Tg)^2}{(g^TBg)(g^TB^{-1}g)}$ and the fact that both $B$ and $B^{-1}$ are positive definite. Thus, as $\alpha$ increases from 0 to 1, $p$ moves along the second segment and $\norm{p}$ keeps growing.
\end{document}
