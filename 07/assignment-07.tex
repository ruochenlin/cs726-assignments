\title{CS 726 Assignment 7}
\author{Ruochen Lin}
\documentclass[11pt]{article}
\usepackage{amsmath,amsfonts,amssymb,amsthm}
\usepackage{mathtools}
\usepackage{commath}
\begin{document}
\maketitle
\section{}
For simplicity, we define $b_k = x_k - x^*$, where we have $\lim_{k\to\infty}b_k=0$. Because
\begin{equation}\begin{split}
\norm{b_k+p_k} = o(\norm{b_k})
\end{split}\nonumber\end{equation} 
and 
\begin{equation}\begin{split}
\norm{b_k+p_k^N} = O(\norm{b_k}^2),
\end{split}\nonumber\end{equation}
we have 
\begin{equation}\begin{split}
p_k-p_k^N &= (b_k+p_k) - (b_k+p_k^N) \\
&= o(\norm{b_k}) - O(\norm{b_k}^2) = o(\norm{b_k}).
\end{split}\nonumber\end{equation} 
In addition, we have
$$\norm{b_k} = O(\norm{p_k}),$$
because otherwise we would have 
\begin{equation}\begin{split} 
\norm{b_k+p_k} &= O\big(\max\{\norm{b_k},\norm{p_k}\}\big) \\
&\geq O(\norm{b_k}), 
\end{split}\nonumber\end{equation} 
which contradicts our condition that $\norm{b_k+p_k} = o(\norm{b_k})$. Thus, we have
\begin{equation}\begin{split}
\norm{p_k-p_k^N} = o(\norm{p_k}).
\end{split}\nonumber\end{equation} 
\section{}
$B_k$ is symmetric positive definite, so it can be diagonalized as $B_k = Q\Lambda Q^T$, with $Q$ being orthogonal and $\Lambda$ being diagonal with only positive nonzero entries. Thus, there exists $B_k^{1/2} = Q\Lambda^{1/2}Q^T$, and the same is true for $B_k^{-1}$. $\mu_k$ can be written in the following form:
\begin{equation}\begin{split} 
\mu_k &= \frac{(y_k^TB_k^{-1}y_k)(s_k^TB_ks_k)}{(y_k^Ts_k)^2} \\
&= \frac{(y_k^TB_k^{-1/2})(B_k^{-1/2}y_k)(s_k^TB_k^{1/2})(B_k^{1/2}s_k)}{(y_k^Ts_k)^2}\\
&\geq\frac{(y_k^TB_k^{-1/2}B_k^{1/2}s_k)^2}{(y_k^Ts_k)^2} \\
&=\frac{(y_k^Ts_k)^2}{(y_k^Ts_k)^2}\\
&=1.
\end{split}\nonumber\end{equation} 

\section{}
If $y_k \neq B_ks_k$ and $(y_k-B_ks_k)^Ts_k=0$, then if there exists $v$ such that $(B_k+\sigma vv^T)s_k = y_k$, $\sigma=\pm1$, then we have
\begin{equation}\begin{split} 
\sigma(v^Ts_k)v = y_k-B_ks_k.
\end{split}\nonumber\end{equation} 
If $v^Ts_k=0$, then
\begin{equation}\begin{split}
(B_k+\sigma vv^T)s_k &= y_k \\
\Rightarrow B_ks_k &= y_k,
\end{split}\nonumber\end{equation} 
which contradicts the condition $y_k \neq B_ks_k$. On the other hand, if $v^Ts_k\neq0$, then $v$ must be a multiple of $y_k-B_ks_k$, which needs to satisfy $(y_k-B_ks_k)^Ts_k=0$; this is contradictory as well. In conclusion, there is no symmetric rank one update satisfying secant equation given the conditions.
\end{document}
