\title{CS 726 Assignment 2}
\author{Ruochen Lin}
\documentclass[11pt]{article}
\usepackage{amsmath,amsfonts,amssymb,amsthm}
\usepackage{mathtools}
\usepackage{commath}
\begin{document}
\maketitle
\section{}
The example in textbook $x_k=x^*+k^{-k}$ decreases to zero Q-superlinearly because \begin{equation}\begin{split}\lim_{k\to+\infty}\frac{\norm{x_{k+1}-x^*}}{\norm{x_{k}-x^*}}&=\lim_{k\to+\infty}\frac{(k+1)^{-(k+1)}}{k^{-k}}=\lim_{k\to+\infty}(1+\frac1k)^{-k}(k+1)^{-1}\\&=\lim_{k\to+\infty}(1+\frac1k)^{-k}\times\lim_{k\to+\infty}\frac1{k+1}=\frac1e\times0\\&=0,\end{split}\nonumber\end{equation} 
in which \begin{equation} \begin{split}\lim_{k\to+\infty} (1+\frac1k)^{-k} &= \lim_{k\to+\infty}\exp(-k\ln(1+\frac1k)) = \exp(-\lim_{k\to+\infty}k\ln(1+\frac1k)\\&=\exp(-\lim_{k\to+\infty} \frac{\ln(1+\frac1k)}{\frac1k}))=\exp(-\lim_{a\to0^+}\frac{\ln(1+a)}a)\\&=\exp(-\lim_{a\to 0^+}\frac{\frac1{1+a}}1)=e^{-1}.\end{split}\nonumber\end{equation} 
Similarly, we can prove ${x_k}$ does not converge to $x^*$ Q-quadratically because \begin{equation}\begin{split}\lim_{k\to+\infty}\frac{\norm{x_{k+1}-x^*}}{\norm{x_{k}-x^*}^2}&=\lim_{k\to+\infty}\frac{(k+1)^{-k-1}}{k^{-2k}}=\lim_{k\to+\infty}(1+\frac1k)^{-k}\frac{(k+1)^{-1}}{k^{-k}}\\&=\lim_{k\to+\infty}(1+\frac1k)^{-k}\frac{k^k}{k+1}=+\infty.\end{split}\nonumber\end{equation}
\section{}
Suppose $i_{max}=\arg\max_{i}\{(\nabla f(x_k))_i\}$ and $d_k$ is given by $$(d_k)_i=-\delta_{i,i_{max}}(\nabla f(x_k))_{i_{max}},$$ then $$\frac{-d_k^T\nabla f(x_k)}{\norm{\nabla f(x_k)}\norm{d_k}}=\frac{\norm{d_k}^2}{\norm{\nabla f(x_k)}\norm{d_k}}=\frac{\norm{d_k}}{\norm{\nabla f(x_k)}}\geqslant\frac1{\sqrt{m}}, $$ with $m$ being the dimensionality of $x$, because the entry passed from $-\nabla f(x_k)$ to $d_k$ is the largest one, and other entries in $\nabla f(x_k)$ cannot exceed the magnitude of this entry. Compare this inequality with the first requirement, we have $\bar{\epsilon}=\frac1{\sqrt{m}}$.\\[0.5cm]
In addition, because $d_k$ is constructed by picking out the largest entry in $-\nabla f(x_k)$, its norm cannot exceed that of $\nabla f(x_k)$, and thus $\frac{\norm{d_k}}{\norm{\nabla f(x_k)}}\leqslant1$. Combine this with our observations from the preceeding part, we have $\gamma_1=\frac1{\sqrt{ m}}$, $\gamma_2=1$.
% We should probably be more cautious about the choice of $\gamma_1$, because our analysis can only guarantee that $\frac{\norm{d_k}}{\norm{\nabla f(x_k)}}$ is no smaller than $\frac1m$; if all entries in $\nabla f(x_k)$ have the same magnitude, then $\norm{d_k}=\frac1m\norm{\nabla f(x_k)}$. Thus a safer choice of $\gamma_1$ might be one that's slightly smaller than $\frac1m$, say $\frac9{10m}$. 

\end{document}
